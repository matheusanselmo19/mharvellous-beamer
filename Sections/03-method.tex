%*----------- SLIDE -------------------------------------------------------------
\begin{frame}[t]{As lideranças das equipes dos Novos Talentos}
    \vspace{0.5cm}
    \begin{columns}
        \column{.01\textwidth}
        \column{.7\textwidth}
            \begin{itemize}
                \item equipe \tikznode{cmark}{RAJA} será liderada por Aziel Freitas
                \item equipe \Circled[outer color=mracula8, inner ysep=8pt]{BORG} será liderada por Mateus Cerqueira.
                \item equipe \Circled[outer color=mracula7, inner ysep=8pt]{BORG} será liderada por Mateus Cerqueira.
                \item equipe \Circled[outer color=mracula9, inner ysep=8pt]{jerotimon} será liderada por Mateus Cerqueira.
                \item equipe TIMON-HM será liderada por Leonardo Lima.
            \end{itemize}
        \column{.29\textwidth}
            \includegraphics[width=.9\textwidth, trim={10cm 0 10cm 0},clip]{equipe}
    \end{columns}
    \vspace{1cm}
    
    \emph{Para este desafio não será cobrado o relatório técnico, porém o acompanhamento deverá seguir o mesmo ritmo dos desafios anteriores.}

    %add circle on word
    \begin{tikzpicture}[remember picture,overlay]
        \draw[mracula5,very thick] (cmark) circle[x radius=8mm,y radius=4mm]; 
    \end{tikzpicture}
%*----------- notes
    \note[item]{Notes can help you to remember important information. Turn on the notes option.}
\end{frame}
%-
%*----------- SLIDE -------------------------------------------------------------
\begin{frame}[t]{O progresso das equipes}
    Um dos indicadores para o acompanhamento das equipes será o percentual de conclusão geral da equipe.
    O planejamento das atividades deverá seguir a metodologia aplicada no desenvolvimento de projetos de robótica.
    \newline
    %\vspace{0.5cm}
    \begin{table}[ht!]
    \centering
        \caption{PERCENTUAL DE CONCLUSÃO POR EQUIPE}
        \begin{tabular}{|l|c|c|c|c|} \hline
            \textbf{EQUIPE}&\textbf{04/05}&\textbf{11/05}&\textbf{18/05}&\textbf{25/05}\\ \hline
            RAJA & 17\% &32\% & &  \\ \hline
            BORG & 0\% &41\% & &  \\ \hline
            TIMON-HM & 5\% &47\% & &  \\ \hline
        \end{tabular}
    \end{table}
%*----------- notes
    \note[item]{Notes can help you to remember important information. Turn on the notes option.}
\end{frame}
%-
%*----------- SLIDE -------------------------------------------------------------
\begin{frame}[t]{O progresso das equipes}
    Um dos indicadores para o acompanhamento das equipes será o percentual de conclusão geral da equipe.
    O planejamento das atividades deverá seguir a metodologia aplicada no desenvolvimento de projetos de robótica.
    \newline
    %\vspace{0.5cm}
    % \begin{table}[ht!]
    % \centering
    %     \caption{PERCENTUAL DE CONCLUSÃO POR EQUIPE}
    %     \begin{tabular}{|l|c|c|c|c|} \hline
    %         \textbf{EQUIPE}&\textbf{04/05}&\textbf{11/05}&\textbf{18/05}&\textbf{25/05}\\ \hline
    %         RAJA & 17\% &32\% & &  \\ \hline
    %         BORG & 0\% &41\% & &  \\ \hline
    %         TIMON-HM & 5\% &47\% & &  \\ \hline
    %     \end{tabular}
    % \end{table}
%*----------- notes
    \note[item]{Notes can help you to remember important information. Turn on the notes option.}
\end{frame}
%-
%*----------- SLIDE -------------------------------------------------------------
\begin{frame}[t]{O progresso das equipes}
    Um dos indicadores para o acompanhamento das equipes será o percentual de conclusão geral da equipe.
    O planejamento das atividades deverá seguir a metodologia aplicada no desenvolvimento de projetos de robótica.
    \newline
    %\vspace{0.5cm}
    % \begin{table}[ht!]
    % \centering
    %     \caption{PERCENTUAL DE CONCLUSÃO POR EQUIPE}
    %     \begin{tabular}{|l|c|c|c|c|} \hline
    %         \textbf{EQUIPE}&\textbf{04/05}&\textbf{11/05}&\textbf{18/05}&\textbf{25/05}\\ \hline
    %         RAJA & 17\% &32\% & &  \\ \hline
    %         BORG & 0\% &41\% & &  \\ \hline
    %         TIMON-HM & 5\% &47\% & &  \\ \hline
    %     \end{tabular}
    % \end{table}
    
    \url{https://braziliansinrobotics.com/}
%*----------- notes
    \note[item]{Notes can help you to remember important information. Turn on the notes option.}
\end{frame}
%-